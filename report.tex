\documentclass[11pt,twocolumn]{article}

% Set margins
\usepackage[margin=0.5in]{geometry}

% Set fonts, both for main and for source code
\usepackage{fontspec}
\setmainfont{"[times_new_roman.ttf]"}
\setmonofont{"[inconsolata.ttf]"}

% formatting for source code
\usepackage{listings}
\usepackage{color}
\definecolor{mygreen}{rgb}{0,0.6,0}
\definecolor{mygray}{rgb}{0.5,0.5,0.5}
\definecolor{mymauve}{rgb}{0.58,0,0.82}
\lstset{
  backgroundcolor=\color{white},   % choose the background color; you must add
                                   % \usepackage{color} or \usepackage{xcolor}
  basicstyle=\ttfamily,            % the size of the fonts that are used for
                                   % the code
  breakatwhitespace=false,         % sets if automatic breaks should only
                                   % happen at whitespace
  breaklines=true,                 % sets automatic line breaking
  captionpos=b,                    % sets the caption-position to bottom
  commentstyle=\color{mygreen},    % comment style
  columns=fullflexible,            % make it wrap columns at max possible
  deletekeywords={...},            % if you want to delete keywords from the
                                   % given language
  escapeinside={\%*}{*)},          % if you want to add LaTeX within your code
  extendedchars=true,              % lets you use non-ASCII characters; for
                                   % 8-bits encodings only, does not work with
                                   % UTF-8
  frame=none,                      % adds a frame around the code, none, single
  keepspaces=true,                 % keeps spaces in text, useful for keeping
                                   % indentation of code (possibly needs
                                   % columns=flexible)
  keywordstyle=\color{blue},       % keyword style
  language=C,                      % the language of the code
  morekeywords={*,...},            % if you want to add more keywords to the
                                   % set
  numbers=none,                    % where to put the line-numbers; possible
                                   % values are (none, left, right)
  numbersep=5pt,                   % how far the line-numbers are from the code
  numberstyle=\color{mygray},      % the style that is used for the
                                   % line-numbers
  rulecolor=\color{black},         % if not set, the frame-color may be changed
                                   % on line-breaks within not-black text (e.g.
                                   % comments (green here))
  showspaces=false,                % show spaces everywhere adding particular
                                   % underscores; it overrides
                                   % 'showstringspaces'
  showstringspaces=false,          % underline spaces within strings only
  showtabs=false,                  % show tabs within strings adding particular
                                   % underscores
  stepnumber=1,                    % the step between two line-numbers. If it's
                                   % 1, each line will be numbered
  stringstyle=\color{mymauve},     % string literal style
  tabsize=2,                       % sets default tabsize to 2 spaces
  title=\lstname                   % show the filename of files included with
                                   % \lstinputlisting; also try caption instead
                                   % of title
}


\newcommand{\NumPackages}{65 }
\newcommand{\NumPatches}{20 }
\newcommand{\NumPatchesAccepted}{5 }


\title{Uncovering Undefined Behavior}
\author{Eric Lubin, Jared Wong}

\begin{document}
\maketitle

\section{Overview}
In this project we study the effects of undefined behavior in open source software
and its ability to cause unstable code to be optimized out by modern compilers.

Given the sheer number of software systems that contain such errors, 
with about 50\% of Debian packages reporting over 80,000 different warnings,
we attempted to begin the process of filtering through these results. 

With the help of STACK\cite{stack}, we analyzed \NumPackages packages, 
from Debian and other locations, and submitted a total of \NumPatches patches, 
of which \NumPatchesAccepted have been accepted thus far, with the rest pending approval.
This paper presents a case study into the common mis-practices of many developers and
has helped us develop a rule-of-humb heuristic as to which types of bugs are more
or less likely to be vulnerabilities.

In section~\ref{sec:threat} we review the threat model, in section~\ref{sec:classification}
we break down the observed bugs into a number of different categories that loosely 
correspond to the the types of bugs STACK can detect, and in section~\ref{sec:conclusion}
we wrap up and discuss future areas of research. Finally, section\ref{sec:appendix} provides an 
appendix of all patches found and submitted thus far.

\section{Threat Model}
\label{sec:threat}
We assume that an attacker has complete knowledge of the source code. Any 
optimization unsafe 

\section{Classification of Undefined Behavior Bugs}
\label{sec:classification}
Generally, the bugs we found due to undefined behavior fell into several
general categories.

\subsection{Null Pointer Dereference}
Null pointer dereference, caused by checks to ensure that pointer is not null,
but pointer is never not null at location. True result optimized out.

\subsection{Signed Integer Operation Overflow}

\subsection{Programmer Error}




\section{Exploits}
Some random exploits

\section{Patches}
Some of our patches

\begin{lstlisting}
n = e - i + 1;  /* number of elements */
if (n <= 0 || !lua_checkstack(L, n))
\end{lstlisting}

\section{Conclusion}
\label{sec:conclusion}
Overall, we have shown the value that STACK brings to the suite of static checkers available to 
developers to verify the correctness and stability of their code. STACK can even present warnings
to the developer that catch simple careless error where certain pointers are not dereferenced, as
was the case with Subversion and Audacity. 

Nonetheless, we have noticed the difficulty with discovering legitimate exploits based on these
undefined behavior bugs. Amidst countless redundant null pointer dereferences, many such bugs
are hidden numerous levels deep from the outwards facing components in these packages. As a result,
the ability to propagate an invalid input into such a bug and then exploit this bug is highly challenging. 

In the future, we hope to take this research further. In particular, we'd like to analyze more large
scale packages such as the llvm, julia, and latex packages. Given the time and processor constraints
of this project and the slight difficulty we had in compiling llvm in the first place for using
with stack, we would like to be able to devote more time into such an endeavor. Furthermore, while we liked
having access to all the Debian packages having already been analyzed by STACK, we spent a lot of our time
building other packages andsuppressing compiler warnings instead of analyzing pstack reports. 
In the past week, we have discussed writing a harness into \texttt{brew install} in order to get STACK
results on a much larger variety of packages. Furthermore, we want to focus on existing bugs to attempt
to continue on our quest for exploitable, public facing bugs..

\bibliographystyle{plain}
\bibliography{report}
\section{Appendix}
\label{sec:appendix}
\end{document}
